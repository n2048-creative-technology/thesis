
\addcontentsline{toc}{chapter}{curiosity}


\begin{center}
\vspace*{\fill}
\Huge curiosity

\vspace{2cm}

\begin{flushright}
\large
\textit{commitment to struggle}
\end{flushright}

\vspace*{\fill}
\end{center}

\normalsize

As many people, I place some of my early memories during my time in elementary school, and similarly to many, such memories are not particularly the most enjoyable ones. It was the time where differences were notorious, misunderstood, and punished. In a german school during the late 80's, where discipline and uniformity appeard as a main value, I learned to defend my position on the left side of this equations: 

curiosity = disobedience
curiosity = commitment to struggle % Deconstructing the status quo against an institutionalized system of meaning making.
curiosity = insubordination

Quoting Aristotle, "all human beings, by nature, desire to know". That was in fact the opening line of his work Metaphysics, highlighting curiosity as a fundamental aspect of human nature.  However, I experienced that curiosity, as a "distracted learning style", is often rejected as a vicious form, as opposed to a virtuous one. Aristotle did actually have an inclination to recommend being studious about one thing (monopragmosyne). Even Plato argued before that curious people suffer from an imbalance in the 3 parts of their soul: reason, spirit and appetite. \citep{perry2020}

It became well established that being curious implies taking risks, failing, making mistakes, "die at least a few times". \citep{foucault1980masked} Foucault reflects on the transformative power of curiosity, suggesting that it involves letting go of established ways of thinking and being open to change, which he metaphorically describes as a form of “dying.”

% I believe the 3dr equality also comes from M. Foucault. maybe from "Discipline and Punish"

Curiosity, in this frame, presents an invitation to explore boundaries and question all norms.
In relation to this, I'll explore unconventional digital mediums, pushing established structures and welcoming the unknown.

The digital and other forms of artwork inspired by this can evolve in forms that resist being fully understood, requiring viewers to engage multiple times or from different perspectives to gain insight, thus embodying a commitment to struggle.
