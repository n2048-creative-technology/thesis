\chapter*{curiosity}
\addcontentsline{toc}{chapter}{curiosity}
\begin{center}
\vspace{2cm}
\begin{flushright}
\large
\textit{commitment to struggle}
\end{flushright}
\vspace*{\fill}
\end{center}
\normalsize

As most people, I place some of my earliest memories in my childhood. It was a time where differences were particularly notorious, misunderstood, and punished. The dictatorship heavily controlled education to align with its ideology, promoting nationalism and suppressing critical thinking. As most children born in this period, I received an education shaped by censorship and limited intellectual freedom. Teachers and curricula avoided topics related to human rights, democracy, or the abuse of the regime. 

I grew up in a society where trust in the government and institutions was deeply eroded. This mistrust certainly influenced my attitude toward authority and civic participation. In a context where discipline and normativity appeared as main values, I learned to defend my position on the right side of this equations: 

curiosity = disobedience

curiosity = insubordination

curiosity = commitment to struggle 

{\scriptsize \textcolor{comment}{\%  Deconstructing the status quo against an institutionalized system of meaning making.}}

"All men by nature desire to know". This is the opening line of Aristotle's Metaphysics, highlighting curiosity as a fundamental aspect of human nature.  However, I experienced that curiosity, as a "distracted learning style", is often rejected as a vicious form, as opposed to a virtuous one. Aristotle had an inclination to recommend being studious about one thing (monopragmosyne). Even Plato argued before that curious people suffer from an imbalance in the 3 parts of their soul: reason, spirit and appetite. \citep{perry2020}

It became well established that being curious implies taking risks, failing, making mistakes, "die at least a few times" \citep{foucault1980masked}. Foucault reflects on the transformative power of curiosity, suggesting that it involves letting go of established ways of thinking and being open to change, which he metaphorically describes as a form of “dying.”

Curiosity, in this frame, presents an invitation to explore boundaries and question all norms. The digital and other forms of artwork inspired by this can evolve in forms that resist being fully understood, requiring viewers to engage multiple times or from different perspectives to gain insight, embodying a commitment to struggle.

The exploration of unconventional media as a way to disrupt the status quo is a recurring theme in media theory. Several theories and philosophical perspectives address this phenomenon. McLuhan's "Understanding Media" \citep{mcluhan1964}, is a good example of this (The medium is the message). Artists using unconventional media are not just creating content, but they are defining new ways to experience and understand such content.

Deleuze and Guatari refer to the idea of deterritorialization, as the process of breaking away from established structures. Their concept of "rhizome" emphasizes non-linear, decentralized forms of thought and creation. \citep{deleuze1980} 

Curiosity drives us to break away from familiar territories, whether intellectual, cultural, or artistic. It encourages us to explore "lines of flight", creating opportunities for new knowledge and experiences. Non-linear, interconnected ways of thinking and being (as a neurodivergent individual), as opposed to hierarchical structures,  allow for an open-ended exploration, where the process is as valuable as the destination.
