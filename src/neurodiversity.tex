

\addcontentsline{toc}{chapter}{neurodivergent}

\begin{center}
\vspace*{\fill}
\Huge neurodivergent

\vspace{2cm}

\begin{flushright}
\large
\textit{The holographic principle suggests that information about a volume of space can be encoded on its boundary, leading to a perspective in which spacetime within that volume, including time, is a projection. Thus, time might not exist as a fundamental property but instead as a result of interactions in this deeper, more fundamental layer of reality.}
\end{flushright}

\vspace*{\fill}
\end{center}

\normalsize

I spent years understanding what this means to me. I learned about the mask I put on unknowingly — to fit in, to attract less attention, to avoid conflicts and misunderstandings. I learned the consequences of wearing this mask. The idea of perception as a controlled hallucination suggests that what we see, know, and understand is no more than the most likely prediction made by our trained brains. A neural network in which an internal conflict arises between an error signal—indicating that what’s in front of us does not match our expectations—and a massively skewed training dataset of memories, insisting that what we know from past experiences is the correct interpretation.

Neurodivergence is now better known and understood, but as a statistical minority, it is not well represented in the dataset of human interactions. It is only logical that it would be difficult to comprehend from the perspective of a neurotypical brain. The issue of skewed datasets is commonly addressed in the context of AI and machine learning. However, while we can design datasets to balance the represented populations, a real brain learns from real interactions, and the statistics remain the same regardless of awareness.

Analytical acceptance, algorithmic forgiveness.

For some neurodivergent individuals, time feels less sequential and more layered or interconnected, as if different "dimensions" of experience coexist and interact simultaneously. Much like a hologram contains a vast amount of information compressed into a simpler form, neurodivergent cognition could compress complex timelines and experiences into non-linear formats, creating unique interpretations and associations across time.
