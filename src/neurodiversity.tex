\chapter*{neurodivergent}
\addcontentsline{toc}{chapter}{neurodivergent}
\begin{center}
\vspace{2cm}
\begin{flushright}
\large
\textit{The holographic principle suggests that information about a volume of space can be encoded on its boundary, leading to a perspective in which spacetime within that volume, including time, is a projection. Thus, time might not exist as a fundamental property but instead as a result of interactions in this deeper, more fundamental layer of reality.}
\end{flushright}
\vspace{2cm}
% \vspace*{\fill}
\end{center}
\normalsize

The idea of perception as a controlled hallucination suggests that what we see, know, and understand is no more than the most likely prediction made by our trained brains. A neural network in which an internal conflict arises between an error signal, indicating that what's in front of us does not match our expectations, and a massively skewed training dataset of memories, insisting that what we know from past experiences is the correct interpretation.

Neurodivergence is now better known and understood, but as a statistical minority, it is not well represented in the dataset of human interactions. It is only logical that it would be difficult to comprehend from the perspective of a neurotypical brain. The issue of skewed datasets is commonly addressed in the context of AI and machine learning. However, while we can design datasets to balance the represented populations, a real brain learns from real interactions, and the statistics remain the same regardless of awareness.

Analytical acceptance, algorithmic forgiveness.

{\scriptsize \textcolor{comment}{\%  Intuition and "the big picture" }}

For some neurodivergent individuals, time feels less sequential and more layered or interconnected, as if different dimensions of experience coexist and interact simultaneously. Much like a hologram contains a vast amount of information compressed into a simpler lower-dimensional form, neurodivergent cognition could compress complex timelines and experiences into non-linear formats, creating unique interpretations and associations across time.
Neurodivergent cognition might operate by projecting internal mental states or processing vast amounts of sensory data into condensed forms like patterns, metaphors, or unique associations. 

The holographic principle challenges the classical idea of locality, suggesting that information can have non-local representations. As a neurodivergent individual, cause-and-effect thinking strategies don't come naturally, favoring lateral connections and holistic insights that reflect non-locality in thought processes. Often a heightened awareness of details, turns into an intuitive grasp of the whole system encoded in parts, as kind of cognitive holography. Most attempts to uncompress this intuition often come across as confusing misunderstandings, since even language is made to reflect linear interpretations of reality through sequential narratives.

"Anything in the territory that resists attempts at modeling thus becomes, in the world of digital models, noise in the system" \citep{benasayag2019}. Benasayag addresses the issue of algorithmic bias, where neural networks may perpetuate existing social prejudices and inequalities. He underscores the need for critical examination of the data and methodologies used in AI to prevent reinforcing discriminatory practices.

{\scriptsize \textcolor{comment}{\%  Exist within the noise }}

RANSAC (RANdom SAmple Consensus) is an iterative algorithm used for estimating the parameters of a mathematical model from a dataset that contains both inliers (data points that fit the model) and outliers (data points that do not fit the model). It is particularly robust and capable of rejecting outliers and is widely used in applications in the presence of noise. We must define new non-probabilistic approaches to social norms and rules that includes outliers, or avoid the models and rules altogether, validating the richness of the full spectrum, avoiding the expectations of coherence to the known set. 

{\scriptsize \textcolor{comment}{\%  in relation to my installation work, and to this text:}}

I'm interested in multi-sensory installations, layered audio-visual compositions, or interactive works that allow viewers to experience various "time slices" of the piece, where events and emotions compress into a single moment. Experiences of layered and non-linear time are certainly an inspiration to an approach that defies linear storytelling or straightforward interaction.

symbiotic contamination

failures with a serial number

In the current context of a neurotypical majority, forging the options and leading to a society that values selection over creativity, the creation of our own tools seems to be an appropriate choice to true creativity. Such practices allow to dig into deeper understanding of the final outcomes, and explore the equally rich properties of every step of a process.   
