\chapter*{wave function collapse}
\addcontentsline{toc}{chapter}{wave function collapse}
\begin{center}
\vspace{2cm}
\begin{flushright}
\large
\textit{ $a \propto E$ }
\end{flushright}
\vspace{2cm}
\end{center}
\normalsize

% This chapter talks about influences of the unknown on anxiety. 
Anxiety is proportional to the entropy of a situation. 

\newpage  
\subsection*{ Entropy, quantum mechanics and puzzles} 

% 
The algorithmic approach to solving a Sudoku puzzle involves identifying the cells with minimum entropy (those with the fewest possibilities remaining). When a possible solution is presented, it restricts the degrees of freedom in adjacent cells, reducing entropy and guiding the puzzle toward resolution. This same principle of information resolution echoes in the realm of quantum mechanics, where a wave function, representing a superposition of possible states, collapses upon observation, resolving a multitude of possibilities into a single, defined outcome.

Every unknown in life, every decision still not made, every unanswered question, extends a branching structure of potential realities, a high-entropy state where all possible futures coexist. In this state, anxiety emerges as a direct consequence of the volume of probabilistic outcomes. When a choice is made, or an observation is recorded, all competing possibilities collapse into one, thereby reducing entropy and, in turn, the existential anxiety associated with the unknown. The resolution of uncertainty into a singular actuality offers a peculiar relief, an end to speculation, but also the loss of alternative futures.

{\scriptsize \textcolor{comment}{\%  Observing as a Constructive Act }}

Observation is never passive.  The Copenhagen interpretation of quantum mechanics tells us that the act of measurement does not reveal a pre-existing state but rather enforces one upon an indeterminate system. The world, then, is not a static, pre-formed entity waiting to be observed, but a participatory system where perception shapes reality. When we observe, we frame, filter, and interpret phenomena through the lens of our preconceptions, cultural codes, and technological mediations. McLuhan suggested, in the context of media ecology, that what we observe is shaped by the tools and contexts of observation. \citep{mcluhan1964}

The act of looking is an active engagement. The technologies used for observation, such as cameras, screens or algorithms, affect the observed object by framing and introducing layers of abstraction, transforming the observer into both a participant and a subject of the observation. Media plays an important role, as it pre-selects and amplifies certain aspects of reality and ignores others, conditioning our gaze.

{\scriptsize \textcolor{comment}{\%  surveillance }}

Michel Foucault extends this perspective into the realm of power and surveillance. The panopticon, a structure in which the awareness of being observed transforms behavior, is a metaphor for the self-regulation imposed by knowledge systems. \citep{foucault1975} The wave function collapse can be read similarly: the moment an entity is observed, its state is determined, and its freedom is cropped. 

%  new
Beyond physics and computation, wave function collapse provides a compelling analogy for decision-making. If each unmade choice represents a superposition of futures, then the moment of commitment enacts a form of collapse, selecting one timeline while discarding all others. Regret, then, can be understood as a longing for collapsed possibilities, an awareness of the alternate selves that could have been.

In contrast, the Many-Worlds interpretation of quantum mechanics suggests that all possible outcomes occur in separate, parallel realities. In this framework, no decision ever truly eliminates an alternative, it merely bifurcates existence into divergent streams.
\\
 
\textit{Engaging with a moment\\Observing regardless of consent\\Collapsing, creating reality}

