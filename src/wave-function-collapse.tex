\chapter*{wave function collapse}
\addcontentsline{toc}{chapter}{wave function collapse}
\begin{center}
\vspace{2cm}
\begin{flushright}
\large
\textit{ $a \propto E$ }
\end{flushright}
\vspace{2cm}
% \vspace*{\fill}
\end{center}
\normalsize

% This chapter talks about influences of the unknown on anxiety. 
Anxiety is proportional to the entropy of a situation. 

\newpage  % Move to the next page
\subsection*{ Entropy, quantum mechanics and puzzles} 

The algorithmic way to solve a sudoku puzzle is to find the cells that present minimum entropy. This means, find the cells where the number of possible options is smaller. When a possible solution is presented to this cell, the cells around them will in turn decrease their entropy. 

According to quantum mechanics, the wave function represents the probabilities of different coexisting realities, that is, until a measurement is made. At the moment of measurement, chance is replaced by actuality. The wave function collapses, and reality is set.

Every unknown in life, every decision still not made, creates a multitude of possibilities, a distribution of parallel potential realities, simultaneously existing in a high entropy state. 

Making a decision, or a discovery, will collapse all possibilities into one, reducing entropy and in consequence reducing the associated anxiety for the unknown. 

{\scriptsize \textcolor{comment}{\%  Observing as a Constructive Act }}

Observation is never passive. When we observe, we frame, filter, and interpret phenomena through the lens of our preconceptions, cultural codes, and technological mediations. McLuhan suggested, in the context of media ecology, that what we observe is shaped by the tools and contexts of observation. 

The act of looking is an active engagement. The technologies used for observation, such as cameras, screens or algorithms, affect the observed object by framing and introducing layers of abstraction, transforming the observer into both a participant and a subject of the observation. Media plays an important role, as it pre-selects and amplifies certain aspects of reality and ignores others, conditioning our gaze.

Michel Fucault discusses how, in relation to the panopticon, observation also defines a relationship between the observer and the observed. The awareness of being watched, creates self-regulation and transforms behavior. {\scriptsize \textcolor{comment}{\%  surveillance }} \citep{foucault1975}


Engaging with a moment

Observing regardless of consent

Collapsing, creating reality

% note: add more theory

