\chapter*{wave function collapse}
\addcontentsline{toc}{chapter}{wave function collapse}
\begin{center}
\vspace{2cm}
\begin{flushright}
\large
\textit{ $a \propto E$ }
\end{flushright}
\vspace{2cm}
% \vspace*{\fill}
\end{center}
\normalsize

% This chapter talks about influences of the unknown on anxiety. 
Anxiety is proportional to the entropy of a situation. 

\subsection*{ Entropy, quantum mechanics and puzzles} 

The algorithmic way to solve a sudoku puzzle is to find the cells 
that present minimum entropy. 
This means, find the cells where the number of possible options is smaller.
When a possible solution is presented to this cell, the cells around them will 
in turn decrease their entropy. 

According to quantum mechanics, the wave function represents the probabilities 
of different coexisting realities, that is, until a 
measurement is made. At the moment of measurement, chance is replaced by 
actuality. The wave function collapses, and reality is set.

Every unknown in life, every decision still not made, creates a multitude of 
possibilities, a distribution of parallel potential realities, simultaneously 
existing in a high entropy state. 

Making a decision, or a discovery, will collapse all possibilities into one, 
reducing entropy and in consequence reducing the associated anxiety for the unknown. 

{\scriptsize \textcolor{comment}{\%  this chapter should also reflect on the act of observing, and its effect on creating realities. }}




