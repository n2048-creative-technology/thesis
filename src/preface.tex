\chapter*{preface}
\addcontentsline{toc}{chapter}{preface}
\normalsize

The title for this thesis comes from an early memory. Without an abundance of organized memories, I still maintain a clear mental image of this corridor. It belongs to a first period of separation from reality, a generator of narratives intended to build sense. Building backwards the parameters for a Markov chain to generate the time sequence of my life events.    

The title sets up a journey of exploration, with a hidden aspect, and the intention of uncovering it. The story ( if there is any ) {r}evolves around what is behind the door, why it is closed, and what it means as a memory.

The corridor leading to the closed door is a path of self-discovery and confrontation of personal fears. It implies a setting that is limiting and confined, adding to an atmosphere of tension lived in Uruguay until the restoration of democracy in 1985. The closed door is the focal point of such tension, always present but unreachable. It represents a barrier between protection and isolation, between reality and imagination.

In this context, I explore the themes of curiosity, the fear of the unknown, and a tendency to be constantly drawn toward the off-limits. What is the fear of the dark if not a fear of the unknown? It’s a driving force for widening the senses and understanding the environment.

This book is the reflection of my current ongoing introspection, an exploration of the negative space of the memory, and an attempt to confront possible pasts in possible realities. As explored by Roland Barthes in his essay "The death of the author"\citep{barthes1967}, it is to be noted that meaning in this thesis arises from intertextual relationships between chapters, not from a self-contained intent.

This book is written in \LaTeX{} (\href{https://www.tug.org/texlive/quickinstall.html}{tug.org}) and managed as code. The written code can be found here: \href{https://github.com/n2048-creative-technology/thesis}{https://github.com/n2048-creative-technology/thesis} 

% Suggestion: Add a brief connecting sentence at the end of the preface to lead into the introduction, highlighting how memory shapes perception and molds our identity.
