
\addcontentsline{toc}{chapter}{Laplace's demon}

\begin{center}
\vspace*{\fill}
\Huge{Laplace's demon}

\vspace{2cm}

\begin{flushright}
\large{
\textit{ \frac{d\textbf{x}}{dt} = f(\textbf{x}, t)}}
\end{flushright}

\vspace*{\fill}
\end{center}


Pierre-Simon de Laplace conceived a thought experiment involving a hypothetical intelligent being with knowledge of the current state of everything and the capacity to process all that information. Under the hypothesis of a deterministic universe, such a being would know both the past and the future, thereby eliminating the perception of time, since everything that exists now would also reveal what was and what will be.

In a much more limited context of both space and time, the constant monitoring of microscopic changes and patterns places me in a position to predict possible futures and assume causality from potential pasts. I live without a normal perception of time, burdened by the overwhelming anxiety of processing all possible realities with the same intensity as the "here and now." Predicting an experience and experiencing the predictions. Presuming a cause for every effect. Speculative remembering, blurring memories.
