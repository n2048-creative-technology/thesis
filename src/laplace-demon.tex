\chapter*{Laplace's demon}
\addcontentsline{toc}{chapter}{Laplace's demon}
\begin{center}
\vspace{2cm}
\begin{flushright}
\large
\textit{$\frac{d\mathbf{x}}{dt} = f(\mathbf{x}, t)$ }
\end{flushright}
\vspace{2cm}
% \vspace*{\fill}
\end{center}
\normalsize

\newpage  % Move to the next page
Pierre-Simon de Laplace conceived a thought experiment involving a hypothetical intelligent being with knowledge of the current state of everything and the capacity to process all that information. Under the hypothesis of a deterministic universe, such a being would know both the past and the future, thereby eliminating the perception of time, since everything that exists now would also reveal what was and what will be.

In a much more limited context of both space and time, the constant monitoring of microscopic changes and patterns places me in a position to predict possible futures and assume causality from potential pasts. I live without a normal perception of time, burdened by the overwhelming anxiety of processing all possible realities with the same intensity as the "here and now." Predicting an experience and experiencing the predictions. Presuming a cause for every effect. 

{\scriptsize \textcolor{comment}{\% Sense belongs to the realm of Aion, not Chronos. }}

Modern physics introduces uncertainty. Quantum mechanics poses that states of matter are probabilistic rather than absolute, breaking the strict determinism of Laplace's vision and the classical Newtonian perspective. Yet, even in a probabilistic universe, the human experience of time remains a construction rather than a fundamental entity.

Deleuze, in "The Logic of Sense," contrasts two modes of time: Chronos, the linear, measurable time of physics, and Aion, the time of pure becoming, where past and future exist in a non-hierarchical relationship.\citep{deleuze1969}

In the context of the digital arts, the idea of predictability often manifests itself in a form that simulates control while embedding elements of randomness and chaos, allowing the viewer to experience the tension between determinism and uncertainty. Generative artworks otfen operate through algorithms to create an aesthetic of deterministic emergence, where each iteration is governed by pre-defined rules yet appears unpredictable to the observer. A perfect example could be the Conway's "\textit{Game of Life}" \citep{wiki:gol}.\footnote{The Game of Life is a cellular automaton created by John Horton Conway in 1970. The evolution evolution of the game is determined only by its initial state, requiring no further input.}

%% image
\begin{figure}
    \centering
    \includegraphics[width=0.8\linewidth]{assets/gol.png} 
    \caption{\small Simkin glider gun - \textit{Conway's Game of Life}.}
    \label{fig:gol}
\end{figure}

Machine learning models, trained on vast amounts of data, function as modern-day deterministic oracles, forecasting human behavior, market fluctuations, and even criminal activity. These systems, however, are not infallible, as they rely on probabilistic statistics rather than absolute determinism. Nonetheless, they shape perception, creating a feedback loop where past behavior is used to constrain future choices.

Social media algorithms, for example, predict and curate content based on prior interactions, effectively scripting a deterministic version of personal experience. The more data fed into these systems, the more precise their predictions, reinforcing a perceived loss of agency. In this context, Laplace's Demon is not an abstract philosophical construct but an active, operational force in digital culture.

Time-based media, such as performance art and interactive installations, challenge determinism by requiring live, unrepeatable participation. These works cannot be fully anticipated or reconstructed, embodying contingency and resisting Laplace's hypothetical absolute knowledge. To resist the determinism imposed by both philosophical constructs and algorithmic systems, art and media practice must continue to foreground unpredictability, contingency, and the indeterminacy of human experience.

%   note: examples ? 
%  Lygia Clark – Developed relational objects that require audience manipulation, resisting predefined artistic outcomes.

% Marina Abramović – Uses audience participation in performances to explore contingency and endurance (The Artist is Present).

% Ryoji Ikeda – Uses generative audiovisual installations that incorporate randomness in real-time (data.scan, A [for 100 Cars]).


Our experience depends on the flow of time, on uncertainty, on the interplay between memory and expectation. Media, art, and technology constantly negotiate between determinism and randomness, constructing and deconstructing the perception of temporal order.


% {\scriptsize \textcolor{comment}{\%  science fiction}}

% Suggestion: Create a bridge to Hypervigilance by discussing how deterministic thinking could exacerbate hypervigilance, as the mind anticipates and reacts to countless possibilities simultaneously.



%  note: add more theory.. couple more pages are needed here 
