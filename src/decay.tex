\addcontentsline{toc}{chapter}{decay}
\begin{center}
% \vspace*{\fill}
\Huge decay
\vspace{2cm}
\begin{flushright}
\large
\textit{ $n \rightarrow p^+ + e^- + \bar{\nu}_e$ }
\end{flushright}
\vspace{2cm}
% \vspace*{\fill}
\end{center}
\normalsize

When an atom has an unbalanced number of protons and neutrons in its nucleus, it becomes unstable. When an element is unstable, it decays. If there are additional neutrons, making the atom heavier and disrupting the internal nuclear forces, a neutron can transform into a proton by emitting an electron and an antineutrino. This type of decay is known as beta-minus decay.

Just like a carbon-14 atom, with an extra pair of neutrons, we carry the weight of indecision, of uncertainty, of forces that throw our lives out of balance. And just like that carbon atom, we decay, emitting electrons and antineutrinos—massless and imperceptible particles we leave behind, transforming. And just like the resulting nitrogen-14, older and stable, we find rest.

In this chapter I explore transformation and impermanence. Glitches and dynamic pieces capable of degrading over time to evolve into new forms. I pay attention to pieces that simulate the decay of (digital) memory and the breakdown of stability.  

