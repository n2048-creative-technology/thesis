\chapter*{hypervigilance}
\addcontentsline{toc}{chapter}{hypervigilance}
\begin{center}
\vspace{2cm}
\begin{flushright}
\large
\textit{Stochastic resonance is a phenomenon in which a signal that is normally too weak to be detected by a sensor can be boosted by adding white noise}
\end{flushright}
\vspace{2cm}
% \vspace*{\fill}
\end{center}
\normalsize

\newpage  % Move to the next page
Whenever I take a walk, I don't just stroll from A to B. I'm constantly monitoring every obstacle, every moving object and person around, everything that can be moved by the wind or shifted by the weight of raindrops. I calculate the next position of every object, adjusting my trajectory to account for the space needed for myself and my companion, when there's one by my side. I walk, and I am in the near future as much as I am in the present—more than most people I've discussed this with.

I observe and analyze the changes in other peoples motion, their patterns and micro-expressions, curiously attempting to predict their intentions, possible thoughts, and probable actions. I play out their next moves in my mind like a game of chess. I'm here and now, yet I am also everywhere before and after. I'm everyone in my own form, simultaneously avoiding and seeking connections.

%  new:
Hypervigilance, characterized by an intensified state of sensory sensitivity and constant scanning for potential threats, often reflects an underlying sense of fear or dread. This heightened alertness is prevalent in individuals with post-traumatic stress disorder (PTSD), where the brain's natural fight-or-flight mechanisms remain overly active long after the traumatic event has passed. Consequently, individuals remain in a chronic state of heightened awareness, perceiving threats even in safe environments.  

{\scriptsize \textcolor{comment}{\% The close door at the end of the corridor comes to mind again. }}

In contemporary society, the omnipresence of media plays a significant role in amplifying this state of hypervigilance. The concept of "Mean World Syndrome," introduced by George Gerbner, suggests that extensive exposure to violent or negative media content can lead individuals to perceive the world as more dangerous than it actually is. This perception fosters a culture of fear, where people take unnecessary precautions against minor or unlikely dangers, while paying less attention to more significant risks. In a digital era where doomscrolling is an unconscious habit, this exposure extends beyond isolated traumatic experiences to a collective, systemic form of hypervigilance. The nervous system, already attuned to threat detection, is primed by an ongoing flood of distressing images and narratives, reinforcing a sense of constant alertness.\citep{wiki:meanworld}
% 
%  new:
This heightened state of awareness is a defining feature of PTSD, where the nervous system remains locked in a state of overdrive long after the initial trauma has passed. But what happens when this hypervigilance extends beyond those directly affected by trauma? Secondary traumatic stress\footnote{Secondary Traumatic Stress (STS), also known as vicarious trauma or compassion fatigue, refers to the emotional distress and psychological symptoms that result from indirect exposure to traumatic events, particularly through the experiences of others.} is becoming more common due to the continuous exposure to distressing media. One does not need to experience war, disaster, or violence firsthand, constant engagement with mediated suffering is enough to induce similar psychological effects. This suggests that hypervigilance is not just an individual response but a cultural condition.
%  note: back " Some researchers suggest.." with real data

{\scriptsize \textcolor{comment}{\%  How could I share a hyper-experience?}}

In their article, Wiesenfeld and Moss emphasizes the counterintuitive role of noise in enhancing signal detection and transmission in nonlinear systems \citep{wiesenfeld1995}. Individuals with ADHD and autism often show heightened response to sensory input which could be seen as a form of "enhanced signal detection". By framing hypervigilance as a system response to noise, this model emphasizes the potential for both challenges and strengths in neurodivergent sensory processing.


%  note:  it's too short. needs more theoretical markers.   I suggest to zoom in on the latent fear or dread that this state of hypervigilance reflects. also people suffering from PTSS have this hypervigilance that seems to be a desease of our times. because you only need to keep updated to what is happening in the world and you run the risk of developing a PTSS disturbance. 

