\chapter*{hypervigilance}
\addcontentsline{toc}{chapter}{hypervigilance}
\begin{center}
\vspace{2cm}
\begin{flushright}
\large
\textit{Stochastic resonance is a phenomenon in which a signal that is normally too weak to be detected by a sensor can be boosted by adding white noise}
\end{flushright}
\vspace{2cm}
% \vspace*{\fill}
\end{center}
\normalsize

\newpage  % Move to the next page
Whenever I take a walk, I don't just stroll from A to B. I'm constantly monitoring every obstacle, every moving object and person around, everything that can be moved by the wind or shifted by the weight of raindrops. I calculate the next position of every object, adjusting my trajectory to account for the space needed for myself and my companion, when there's one by my side. I walk, and I am in the near future as much as I am in the present—more than most people I've discussed this with.

I observe what everyone else sees, and I analyze the changes in their motion patterns and facial expressions, curiously attempting to predict their intentions, possible thoughts, and probable actions. I play out their actions in my mind like a game of chess. I'm here and now, yet I am also everywhere before and after. I'm everyone in my own form, simultaneously avoiding and seeking connection.

The brain's "signal detector" operating in an overly sensitive state, amplifies both real and perceived threats. Constant monitoring, responsiveness, attention to subtle changes, amplified details that go often unnoticed. 

{\scriptsize \textcolor{comment}{\%  How could I share a hyper-experience?}}

In their article, Wiesenfeld and Moss emphasizes the counterintuitive role of noise in enhancing signal detection and transmission in nonlinear systems \citep{wiesenfeld1995}. Individuals with ADHD and autism often show heightened response to sensory input which could be seen as a form of "enhanced signal detection". By framing hypervigilance as a system response to noise, this model emphasizes the potential for both challenges and strengths in neurodivergent sensory processing.

%  note:  it's too short. needs more theoretical markers.   I suggest to zoom in on the latent fear or dread that this state of hypervigilance reflects. also people suffering from PTSS have this hypervigilance that seems to be a desease of our times. because you only need to keep updated to what is happening in the world and you run the risk of developing a PTSS disturbance. 

