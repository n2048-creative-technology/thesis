\chapter*{memory}
\addcontentsline{toc}{chapter}{memory}
\begin{center}
\vspace{2cm}
\begin{flushright}
\large
% \textit{ some quote here}\\
% \textbf{ attribution }
\end{flushright}
\vspace{2cm}
% \vspace*{\fill}
\end{center}
\normalsize

The military dictatorship in Uruguay that started in 1973 finally came to its end in 1985. By that time, I was 5 and carefully kept away from all the struggles and terrors that happened during that period. Some human experiences can't easily be described in words, but perhaps those words are not even needed. Even though I have no personal memories of the dictatorship itself, the societal impacts of the regime had a significant influence. The door at the end of the corridor kept us in the dark, perhaps protected from ideas that I couldn't even attempt to grasp. Sounds were replaced with silence and the sleeping time was replaced with long stares at fluorescent constellations and doodles of imaginative inventions.    

I can't avoid creating evolving narratives that reflect the fluidity of memory itself. 

Many families of the children born around the 1980s were deeply affected by state repression. Parents who were political dissidents, union activists, or simply suspected of opposing the regime often faced imprisonment, torture, or exile. If not the near family, friends of any close connection to this situations would affect the dynamics of tension and increased anxiety. To protect children, families often avoided discussing politics, creating an atmosphere of silence and fear. Children absorbed the lingering trauma of parents who had suffered under the dictatorship. This trauma could manifest in overprotectiveness, anxiety, or suppressed anger in family dynamics.

The concept of "speculative remembering", where memories blur and predictions merge with present experiences plays a role in creating an "all-knowing" archive that adapts over time \citep{dutt2024}. In theid 2024 article "The speculative memory: contextualising memory in speculative fiction" the authors enphasize how memory underpins personal identity by shaping narratives of self, as well as the ways in which traumatic memories disrupt perceptions of reality and identity. 

Lean-Luc Godard’s film "Here and Elsewhere" (Ici et Ailleurs, 1976) touches the themes of representation and history and reflects on the the political memory of images and the ways in which a non-linear and fragmented memory can be reassembled in different ways based on the context. The film questions the ethics of remembering through images, questioning the reduced representation of a true past. 

This thesis is too an invitation to become more critical about our own processes of remembering, and how memory is shaped by media and context. It's important to note how personal and collective histories are remembered, forgotten and rewritten over time.

Memory behaves sometimes as an interactive installation, capable of recalling previous viewer interactions, layering them as part of the piece, altering and separating it from its original self. 

In Camera Lucida, Roland Barthes distinguishes between the studium (the cultural, intellectual response to an image) and the punctum (its personal, emotional impact). He reflects on the role of the viewer in the construction of meaning. The memory of a closed door, the need for bridging the unknown with rational narratives, the context of my own neurodivergent experience. Constructing meaning.  \citep{barthes1993}

% maybe make some remarks about forms of social amnesia caused by media ? 

% A mutable and subjective phenomenon, invites an exploration of how digital media captures, stores, and alters information.



