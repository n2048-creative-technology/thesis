
\addcontentsline{toc}{chapter}{perception}

\begin{center}
\vspace*{\fill}
\Huge perception

\vspace{2cm}

\begin{flushright}
\large
\textit{Self-Organized Criticality}
\end{flushright}

\vspace*{\fill}
\end{center}

\normalsize

Loud drones, low frequency soothing sounds.
Whispers louder than the loudest screams. 
A new detail that changed my day. 
The repetitive, unsettling  touch.
Tight knotts like hugs. 
Invasive gazes that were not supposed to last.
The faces, the mirrors, the shadows. 


Self-Organized Criticality describes how certain systems naturally evolve toward a critical, highly sensitive state where small changes can lead to large-scale effects. This state of criticality is "self-organized" because the system doesn’t require external tuning to reach this point. It naturally arranges itself into this state through its own dynamics. These models help describe the experience of sensory amplification, where the world can be perceived in vivid detail or with overwhelming intensity. 