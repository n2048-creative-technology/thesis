\chapter*{introduction}
\addcontentsline{toc}{chapter}{introduction}
\begin{center}
\vspace{2cm}
\begin{flushright}
\footnotesize 
\lstinputlisting[language=Python]{quine.py}
\end{flushright}
\vspace{2cm}
\end{center}
\normalsize

% talk about molds.
There is stubbornness in the craft of casting materials through mold making, despite how rewarding it can be. Its whole process makes it hard to allow for later changes. The mold is not the memory of a piece, nor its essence, but it will define its final shape. Is the environment in which we grow and develop ourselves such a kind of mold? 

I remember only fragments about my own past, but I've spent the last few years making stronger efforts to understand the ways in which I perceive my own "umwelt", why I react, and what I react to.  What shaped this current way of thinking? Without an objective memory of my own history, creating versions of this multidimensional mold in which I've cast my way of perceiving has become an iterative process of re-creation.

{\scriptsize \textcolor{comment}{\% recursive alterations allow for a progressive reshaping of perception. }}

The small snippet of code at the start of this page is called a Quine. It is a program that produces its own source code as output, exemplifying a form of computational self-reference. 

{\scriptsize \textcolor{comment}{\% The quine and implies a connection between software, computer models and a human tendency for self-replicating based on our current understanding of reality.}}

Gödel's incompleteness theorem proves that any formal system capable of expressing arithmetic contains statements that refer to themselves cannot be proven true or false within the system. It shows that a self-referential system cannot demonstrate its own consistency.

Perhaps the notion of a quine, or of a self-referential system, relates to the idea of creating our own model of the world, and the difficulty of interpreting the reality as something different than the one that is predefined in our brains. 

% It also reflects on the boundaries between artworks and the generative systems.

