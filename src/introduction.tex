\addcontentsline{toc}{chapter}{introduction}

\begin{center}
\vspace*{\fill}
\Huge introduction

\vspace{2cm}

\begin{flushright}
\footnotesize 
\begin{verbatim}
 w = 'w = {}{}{}; print(w.format(chr(39), w, chr(39)))'; print(w.format(chr(39), w, chr(39)))
\end{verbatim} 
\end{flushright}
\vspace*{\fill}
\end{center}

\normalsize

% talk about molds.

I find stubbornness in the craft of casting materials through mold making, despite how rewarding it can be. The whole process makes it hard to allow for later changes. The mold is not the memory of a piece, nor its essence, but it will define it's final shape. Is the environment in which we grow and develop ourselves such a kind of mold? 

I remember very little about my own past, but I’ve spent the last few years making stronger efforts to understand the ways in which I perceive my own "umwelt", why I react, and what I react to. What shaped this current way of thinking? Without an objective memory of my own history, creating versions of this multidimensional mold in which I’ve cast my way of perceiving has become an iterative process of re-creation, perhaps allowing the casting of new materials.

% recursive alterations allow for a progressive reshaping of percepion. 

% The quine and implies a connection between software, computer models and a human tendency for self-replicating based on our current understanding of reality.

The little snippet of code at the start of this page is called a Quine. It is a special type of program that will output its own code when executed. 
This connects to the idea of creating our own model of the world, and the difficulty of interpreting the reality as something different than the one that is predefined by the observer. It suggests a connection to artworks and processes that are self-referential, and reflects on the boundaries between the artwork and the system generating it.


