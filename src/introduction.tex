
\chapter*{Introduction}
\addcontentsline{toc}{chapter}{Introduction}

% \section*{Motivation}

% i'm already disapointed 

I find stubbornness in the craft of casting materials through mold making, despite how rewarding it can be. The whole process makes it hard to allow for later changes. The mold is not the memory of a piece, nor its essence, but it will define it's final shape. Is the environment in which we grow and develop ourselves such a kind of mold? 

I remember very little about my own past, but I’ve spent the last few years making stronger efforts to understand the ways in which I perceive my own "umwelt", why I react, and what I react to. What shaped this current way of thinking? Without an objective memory of my own history, creating versions of this multidimensional mold in which I’ve cast my way of perceiving has become an iterative process of re-modelling, perhaps allowing the casting of new materials.


\footnotesize \begin{verbatim}
w = 'w = {}{}{}; print(w.format(chr(39), w, chr(39)))'; print(w.format(chr(39), w, chr(39)))
\end{verbatim} \normalsize

% \subsection{Problem Statement}


% \subsubsection{Research Questions}
