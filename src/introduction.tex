\chapter*{introduction}
\addcontentsline{toc}{chapter}{introduction}
\begin{center}
\vspace{2cm}
\begin{flushright}
\footnotesize 
\lstinputlisting[language=Python]{quine.py}
\end{flushright}
\vspace{2cm}
\end{center}
\normalsize

\newpage  % Move to the next page
% talk about molds.
There is stubbornness in the craft of casting materials through mold making, despite how rewarding it can be. Its whole process makes it hard to allow for later changes. The mold is not the memory of a piece, nor its essence, but it will define its final shape. Is the environment in which we grow and develop ourselves such a kind of mold? 

I remember only fragments about my own past, but I've spent the last few years making stronger efforts to understand the ways in which I perceive my own "umwelt", why I react, and what I react to.  What shaped this current way of thinking? Without an objective memory of my own history, creating versions of this multidimensional mold in which I've cast my way of perceiving has become an iterative process of re-creation.

{\scriptsize \textcolor{comment}{\% recursive alterations allow for a progressive reshaping of perception. }}

The small snippet of code under the title of this chapter is called a Quine. It is a program that produces its own source code as output, exemplifying a form of computational self-reference. 

{\scriptsize \textcolor{comment}{\% The quine and implies a connection between software, computer models and a human tendency for self-replicating based on our current understanding of reality.}}

Gödel's incompleteness theorem proves that any formal system capable of expressing arithmetic contains statements that refer to themselves cannot be proven true or false within the system. It shows that a self-referential system cannot demonstrate its own consistency. This means that any attempt at complete self-referential closure inevitably leads to undecidability or incompleteness, as there are always truths outside the system's ability to demonstrate them. 

Yuk Hui's \textit{Recursivity and Contingency} \citep{Hui2019} explores the relationship between technical systems, philosophy, and computational logic. He describes recursivity as a form of self-referentiality in technical, biological, and philosophical systems. However, Hui introduces contingency as the space for unexpected possibilities and alternative configurations beyond purely deterministic, recursive closure. His idea of contingency refers to the openness, indeterminacy, and possibility of alternatives beyond deterministic or purely recursive systems. Contingency interrupts recursion, allowing for emergence and transformation.

Gödels theorem resonates with Hui's argument in that recursion alone does not guarantee self-sufficiency. Systems require contingency to evolve beyond a rigid self-reference. Gödel's results problematize deterministic, computationalist views of reality, aligning with Hui's critique of purely recursive structures in cybernetics.

Hui's philosophical contingency implies that no system can fully determine its own future. There is always the possibility of disruption, reinterpretation, and reconfiguration. Openness, creativity, and evolution require the ability to break out of purely recursive structures.

Perhaps the notion of a quine, or of a self-referential system, relates to the idea of creating our own model of the world, and the difficulty of interpreting the reality as something different than the one that is predefined in our brains. 

% {\scriptsize \textcolor{comment}{\% This intro is not an intro, as the chapters that follow are not chapters.  }}
\footnotesize 
\begin{tcolorbox}[colback=gray!20, colframe=black, arc=2mm, boxrule=0.8pt]\lstinputlisting[language=Python]{quine2.py}
\end{tcolorbox}
\normalsize