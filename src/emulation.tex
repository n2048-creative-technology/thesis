\chapter*{emulation}
\addcontentsline{toc}{chapter}{emulation}
\begin{center}
\vspace{2cm}
\begin{flushright}
\large
\textit{ Human beings are creatures who practice and train, creatures who are free to reach beyond themselves in the process of becoming.}\\
\textbf{ Peter Sloterdijk } 
\citep{sloterdijk2014}
\end{flushright}
\vspace{2cm}
% \vspace*{\fill}
\end{center}
\normalsize

I spent years understanding what this means to me. I learned about the mask I put on unknowingly — to fit in, to attract less attention, to avoid conflicts and misunderstandings. I learned the consequences of wearing this mask. 

Living often feels like running a sophisticated emulation program on a computer. On the surface, the emulated environment mimics a typical operating system, seamlessly performing tasks and following expected protocols. However, behind this facade of normality, a complex system is working overtime to replicate behaviors and responses that comes naturally to others. Constantly striving to appear organized, focused, and in control, while battling with distraction, impulsivity, and a torrent of unfiltered thoughts.

Just as an emulated system can lag or crash when overloaded, I become overwhelmed and fatigued by the continuous effort to conform to neurotypical standards. The emulation requires immense mental resources, leading to burnout and a sense of disconnection from my authentic self.

This section questions the boundaries between imitation and authenticity. Just as in an imperfectly emulated operating system, deeper layers can only be revealed by interaction.

