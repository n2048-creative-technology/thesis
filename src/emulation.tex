\chapter*{emulation}
\addcontentsline{toc}{chapter}{emulation}
\begin{center}
\vspace{2cm}
\begin{flushright}
\large
\textit{Human beings are creatures who practice and train, creatures who are free to reach beyond themselves in the process of becoming.}\\
\textbf{Peter Sloterdijk} \citep{sloterdijk2014}
\end{flushright}
\vspace{2cm}
% \vspace*{\fill}
\end{center}
\normalsize

\newpage  % Move to the next page
I learned about the mask I put on unknowingly to fit in, to attract less attention, to avoid conflicts and misunderstandings. I learned the consequences of wearing this mask.

Living often feels like running an emulation program, replicating behaviors and responses that come naturally to others. On the surface, the emulated environment mimics a typical operating system, seamlessly performing tasks and following expected protocols. 

The effort to conform to neurotypical standards can be exhausting, often overwhelming and disconnecting.

The tension between imitation and authenticity mirrors the challenges of emulation. Like an emulator replicating the functionality of another system, masking often relies on recombining observed behaviors to navigate social environments. But emulation, by its nature, exposes the limits of replication, revealing deeper truths through interaction and engagement.

In his essay \textit{The Work of Art in the Age of Mechanical Reproduction}, Walter Benjamin describes the uniqueness of a piece as its \textit{aura}, and argues that mechanical reproduction diminishes the aura of an original work of art, affecting its authenticity \citep{benjamin1935}. 

The rise of digital art and AI technologies further complicates the discussion on authenticity. In the digital realm, the ease of replication and distribution encourages a reevaluation of authenticity. The available tools contradict, or at the very least challenge the traditional criteria for what constitutes an original piece, and the notion of what adds value. Such technological developments reveal the ideological function of authenticity, demonstrating how claims to originality often rely on exclusionary and artificial distinctions between "real" and "fake."

Large language models, or \textit{LLMs} offer a powerful example of emulation, imitating human-like language and creativity while simultaneously challenging traditional notions of originality and authenticity.

% note: 2 sources: " Adorno - The Jargon of Authenticity" in which he critiques the false notion about authenticity and the reactionary narrative that is related to it. (Emily Bernstein's thesis of last year) and the notion of algorithmic authenticity (chun et al)

% NEW %%%%%%%%%%%%%%%%%%%%%%%%%%5555
\citep{adorno1973}. 

The discourse of authenticity has long been central to discussions of artistic and cultural production, often serving as a measure of originality and creative legitimacy. However, as Theodor W. Adorno critiques in \textit{"The Jargon of Authenticity"} \citep{adorno1973}, the very notion of authenticity is frequently employed as an ideological construct, one that obscures rather than reveals the underlying conditions of production. This critique is particularly relevant in the context of emulation, where the replication of form and function challenges the perceived dichotomy between originality and imitation. He argues that the insistence on being "true to oneself" or finding an "authentic" mode of being is a way of internalizing and naturalizing systems of control rather than engaging in meaningful critique or transformation.

This critique extends to cultural production, where the valorization of authenticity often serves to exclude certain forms of imitation or replication. The fetishization of originality masks the ways in which all cultural artifacts are socially and historically mediated, constructed through networks of influence, appropriation, and reinterpretation. In this sense, emulation, rather than being a deviation from authenticity, exposes the very contingency of originality itself.


LLMs are trained on gigantic datasets of text. They calculate probabilities of possible words and generate text that mimics human communication, often indistinguishable from content created by real people. LLMs emulate linguistic styles, cultural idioms, and intellectual processes by identifying patterns in existing data. 

Through prompt engineering, users collaborate with LLMs to refine outputs, curating their emulation capabilities. This interplay demonstrates how AI tools can enhance human creativity while raising questions about the nature of authenticity. If authenticity lies in human origin, AI lacks it. But if authenticity is somehow measured based on the audience's experience, then LLM-generated texts can feel authentic, even when created by non-human systems.

Deeper connections can be found between AI emulation and human cognition. Both systems recombine existing information to create something new. The distinction lies in intent: human thought is driven by curiosity, emotions, and purpose, which add layers of meaning that is (currently) absent in AI's mechanical processes.
% %%%%%%%%%%%%%%%%%%%%%%%%%%%%%

{\scriptsize \textcolor{comment}{\%  biological algorithms}}

Masking similarly involves effortful emulation. By adapting to environments shaped by neurotypical norms, masking suppresses natural tendencies to conform to expected patterns. This raises questions about the boundaries between imitation and authenticity in human interactions. Am I authentic if my responses are carefully curated and contextually appropriate but lack an intrinsic connection to the emulated behaviour?

Ultimately, whether in art, technology, or human behavior, the interplay between imitation, emulation, and authenticity challenges us to consider the deeper meanings behind our actions and the layers of intent that define who we are.

By applying Adorno's critique of authenticity to the concept of emulation, we can move beyond simplistic binaries of original versus copy, authentic versus inauthentic. Rather than lamenting the supposed erosion of authenticity in a world increasingly mediated by digital reproduction, we might instead embrace emulation as a mode of critical engagement, one that reveals the complex interplay of influence, repetition, and transformation inherent in all creative acts.