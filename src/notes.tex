

General feedback: 

Expand 

hypervigilance
laplace's demon
wave function collapse
emulation
decay

Explain why my interests are relevant to the thesis:  
 - why rationality and logic?
 - why physics and algorithms? 

 
 #Intro (or preface)
 explain the use of comments 


 

## Neurodivergence

explain  "Analytical acceptance, algorithmic forgiveness." 


For some neurodivergent individuals, time feels less
sequential and more layered or interconnected, as if different
dimensions of experience coexist and interact simultaneously.

Give an example from my bio or literature. 

From Benasayang2019, expand on the need to problematize "locality". 


"time slices"

examples of rtwork: 

# Hiroshi Sugimoto – Theater Series (1975–ongoing)
Long-exposure photographs of cinemas where an entire film is projected onto a single frame, collapsing a full-length movie into a single luminous image. This work presents time not as linear but as an accumulation of all its moments at once.

https://yoshiigallery.com/exhibition/hiroshi-sugimoto-theaters-1978-1993


# Rafael Lozano-Hemmer – 33 Questions per Minute (2000)
A generative installation that displays a rapidly changing sequence of randomly generated questions on LCD screens, exceeding the speed at which they can be read. This creates a layered simultaneity of potential meanings and missed moments in time

https://www.lozano-hemmer.com/33_questions_per_minute.php



#David O’Reilly – Everything (2017, interactive simulation)

A digital ecosystem where players can experience multiple temporal scales at once, shifting between perspectives of individual atoms, animals, and galaxies, revealing time as a simultaneous, interconnected whole.

https://en.wikipedia.org/wiki/Everything_(video_game)

These works emphasize non-linear time, simultaneity, and algorithmic processes, creating temporal experiences that challenge traditional storytelling and perception. Would you like examples that focus on a specific type of digital media (e.g., VR, AI, generative art)?


