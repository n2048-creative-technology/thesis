


introduction:

self-referential forms of artwork
(Gödel)

curiosity:

explore unconventional digital mediums (as a way to defy norms)
deleuze, mcLuhan

Neurodibvergent:

find artists / philosophers that explore the topic of squeued datasets 


laplace's daemon:

research works that explore 
- randomness 
- chaos 
- unpredictability 

hypervigilance

explore ways to share a hyperexperience
 

perception: --> this chapter is very short!!! add more

explore artworks that amplify perception 
(sound installations)

decay: --. this is now the final chapter... make it longer 


explore decay in artworks 
 - glitch and degradation 


 memory: 

% Suggestion: Consider interweaving memory earlier, as it’s a foundational theme that resonates throughout the thesis. For example, introduce it after the introduction to set the stage.


General notes: 

Some chapters, like "Laplace’s Demon" and "Hypervigilance," are more academic, while others, such as "Curiosity," are narrative. This tonal shift might confuse readers.